% !TEX TS-program = xelatex
% !TEX encoding = UTF-8 Unicode

\documentclass[11pt]{article}
\usepackage[letterpaper, margin=1in, includehead, includefoot]{geometry}

\usepackage{enumitem}

\usepackage{fontspec}
\setmainfont{Times New Roman}

\usepackage{hyperref}
\hypersetup{
	colorlinks   = true, %Colours links instead of ugly boxes
	urlcolor     = blue, %Colour for external hyperlinks
	linkcolor    = black, %Colour of internal links
	citecolor   = black %Colour of citations
}



\usepackage{datetime}

\newdateformat{monthyeardate}{%
  \monthname[\THEMONTH], \THEYEAR}

\usepackage{hyperref}

\newcommand{\asof}{\monthyeardate\today}

\usepackage{lastpage}

\usepackage{fancyhdr}
\pagestyle{fancy}
\setlength{\headheight}{12pt}
\renewcommand{\headrulewidth}{0pt} % decorative line below header
\setlength{\headsep}{15pt}
\cfoot{Page {\thepage} of {\pageref*{LastPage}}} % page number
\fancyhead[L]{Eunsun Jou}
\fancyhead[R]{\asof}
\fancypagestyle{firstpage}{%
	\lhead{}
	\rhead{\asof}
}


\usepackage{multirow}

\setlength\parindent{0pt}

\newcommand{\sect}[1]{{\fontsize{15}{25}\selectfont \textbf{#1}} {\vspace{0.1cm}} \hrule {\vspace{0.3cm}}}

\newcommand{\subsect}[1]{{\fontsize{12}{18}\selectfont \textit{\textbf{#1}}} {\vspace{0.3cm}}}

\begin{document}

%%%%%%%%%%%% TITLE %%%%%%%%%%%%%
\begin{center}
{\Large \textbf{Eunsun Jou}}
\end{center}
%%%%%%%%%%%%%%%%%%%%%%%%%%%%%

\thispagestyle{firstpage}

%%%%%%%%%%%% CONTACT %%%%%%%%%%%%%
\sect{Contact}

\begin{minipage}[t]{0.5\textwidth}
\begin{flushleft}
\href{mailto:jou@mit.edu}{jou@mit.edu}

\href{https://eunsunjou.github.io}{https://eunsunjou.github.io}
\end{flushleft}
\end{minipage}
\begin{minipage}[t]{0.5\textwidth}
\begin{flushright}
MIT Department of Linguistics and Philosophy

77 Massachusetts Avenue, 32D-866

Cambridge, MA, USA 02139
\end{flushright}
\end{minipage}

%%%%%%%%%%%%%%%%%%%%%%%%%%%%%%%%

\vspace{1cm}

%%%%%%%%%%%% EDUCATION %%%%%%%%%%%%%
\sect{Education}

\subsubsection*{Degree Programs}
\begin{tabular}{p{0.1\textwidth}|p{0.9\textwidth}}
    {2024}&{\textbf{Ph.D. in Linguistics -- Massachusetts Institute of Technology}}\\
    {(expected)}&{}\\
\end{tabular}

\vspace{0.2cm}

\begin{tabular}{p{0.1\textwidth}|p{0.9\textwidth}}
	{2018}&{\textbf{M.A. in Linguistics -- Seoul National University}}\\
 \end{tabular}

\vspace{0.2cm}

\begin{tabular}{p{0.1\textwidth}|p{0.9\textwidth}}
   {2016}&{\textbf{B.A. in Linguistics -- Seoul National University}}\\
\end{tabular}

\subsubsection*{Non-Degree Programs}
\begin{tabular}{p{0.1\textwidth}|p{0.9\textwidth}}
{2022}&{{\textbf{Crete Summer School of Linguistics.}} Rethymno, Greece.}\\
{2017}&{\textbf{Linguistic Institute -- Linguistic Society of America.} Lexington, KY, USA.}\\
\end{tabular}

%%%%%%%%%%%%%%%%%%%%%%%%%%%%%%%%

{\vspace{1cm}}

{\sect{Publications}}

{\subsect{Articles}}

2024. Honorification as Agree in Korean and Beyond. {\textit{Glossa: a journal of general linguistics}} 9(1). {\href{https://doi.org/10.16995/glossa.9565}{https://doi.org/10.16995/glossa.9565}}\\

{\subsect{Conference Proceedings}}

{To appear.  Positional restriction on case assignment: Evidence from Korean nominal adverbials. {\textit{Proceedings of WAFL 17}.}}\\

{2023. An economy-based amendment to learning hidden structure with Robust Interpretive Parsing. \textit{Supplemental Proceedings of the 2022 Annual Meeting on Phonology.}}\\

{2018. Embedded topicalization in Korean factive complement clauses: An experimental approach. \textit{Proceedings of Japanese/Korean Linguistics 26.}}\\

{2017. The distribution and interpretation of the Korean topic marker \textit{nun} -- a feature-inheritance approach. \textit{Proceedings of the 2017 Linguistic Society of Korea Summer Conference}. 145-156\\

{\sect{Unpublished Manuscripts}}}

2018. Embedded Topicalization in Korean Factive Complement Clauses: An Experimental Approach. M.A. Thesis. Seoul National University.

%%%%%%%%%%%%%%%%%%%%%%%%%%%%%%%%

{\vspace{1cm}}

{\sect{Presentations}}

{2023. Positional restriction on case assignment: Evidence from Korean nominal adverbials. Workshop on Altaic Formal Linguistics 17 (WAFL 17).}\\

{2023. Case marking of Korean nominal adverbials correlates with subject position. The 25th Seoul International Conference on Generative Grammar (SICOGG 25).}\\

{2023. Case marking of Korean nominal adverbials correlates with subject position. Syntax seminar at Seoul National University. }\\

{2022. An economy-based amendment to Robust Interpretive Parsing with the GLA. Annual Meeting on Phonology (AMP 2022).}\\

{2022. Korean addressee honorification as Cyclic Agree at Force: Comparison with Magahi. The 45th Generative Linguistics in the Old World (GLOW 45).}\\

{2019. Embedded topicalization in Korean factive complement clauses. 93rd Annual Meeting of the Linguistic Society of America (LSA).}\\

{2018. Embedded topicalization in Korean factive clauses and islands: Experimental approach. The 26th Japanese/Korean Linguistics Conference (J/K 26).}\\

{2017. The distribution and interpretation of the Korean topic marker \textit{nun} -- a feature-inheritance approach. The Linguistic Society of Korea Summer Conference.}

{\vspace{1cm}}

\sect{Teaching Assistantships}

Spring 2023. Introduction to Linguistics (MIT 24.900)

{\hphantom{Spring 2023.}} Instructor: Donca Steriade\\

Summer 2022. Introduction to Phonology (Crete Summer School of Linguistics)

{\hphantom{Summer 2022.}} Instructor: Douglas Pulleyblank\\

Spring 2022. Language and Structure II: Syntax (MIT 24.902)

{\hphantom{Spring 2022.}} Instructor: Sabine Iatridou\\

Fall 2021. Introduction to Linguistics (MIT 24.900)

{\hphantom{Fall 2021.}} Instructor: Adam Albright\\

\sect{Services}

\subsect{Academic Services}

\vspace{-0.75cm}

\subsubsection*{Ad-hoc reviewing for journals}

Linguistic Inquiry (2023), Journal of East Asian Linguistics (2024)

\subsubsection*{Conference reviews}

NELS (2023), CLS (2024)

\vspace{\baselineskip}

\subsect{Departmental Services}

\vspace{-0.75cm}

\subsubsection*{Massachusetts Institute of Technology}
Fall 2021 -- Spring 2022. Colloquium co-organizer

Fall 2020 -- Spring 2021. Syntax Square (Reading group) co-organizer

\subsubsection*{Seoul National University}
{2017. Research Assistant for the World-leading University Fostering Program.

\begin{itemize}[leftmargin=15pt, topsep=0pt, itemsep=0pt, parsep=0pt]
	\item{{\small Assistant to organizing committee of the 19th Seoul International Conference on Generative Grammar (SICOGG)}}
	\item{{\small Collected and compiled data for the departmental consulting project}}
\end{itemize}
}

\vspace{1cm}

\sect{Scholarships}

\textbf{First Year Presidential Fellowship} (2019)\\{Massachusetts Institute of Technology}\\

{\textbf{Eminence Scholarship}} (2015)\\{Seoul National University}\\

\vspace{1cm}

\sect{Skills}

\subsubsection*{Languages}
\noindent{Korean (native)}, English (fluent), French (advanced), German (reading knowledge)
\subsubsection*{Computational}
\noindent{Python,} {\LaTeX}, R, Perl

\end{document}
%%% Local Variables:
%%% mode: latex
%%% TeX-master: t
%%% End:
