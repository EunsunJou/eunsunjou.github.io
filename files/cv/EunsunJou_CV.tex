% !TEX TS-program = xelatex
% !TEX encoding = UTF-8 Unicode

\documentclass[12pt]{article}
\usepackage[letterpaper, margin=1in, includehead, includefoot]{geometry}

\usepackage{enumitem}

\usepackage{fontspec}
\setmainfont{Times New Roman}

\usepackage{hyperref}
\hypersetup{
	colorlinks   = true, %Colours links instead of ugly boxes
	urlcolor     = blue, %Colour for external hyperlinks
	linkcolor    = black, %Colour of internal links
	citecolor   = black %Colour of citations
}



\usepackage{datetime}

\newdateformat{monthyeardate}{%
  \monthname[\THEMONTH], \THEYEAR}

\newcommand{\asof}{\monthyeardate\today}

\usepackage{lastpage}

\usepackage{fancyhdr}
\pagestyle{fancy}
\setlength{\headheight}{12pt}
\renewcommand{\headrulewidth}{0pt} % decorative line below header
\setlength{\headsep}{15pt}
\cfoot{Page {\thepage} of {\pageref*{LastPage}}} % page number
\fancyhead[L]{Eunsun Jou}
\fancyhead[R]{\asof}
\fancypagestyle{firstpage}{%
	\lhead{}
	\rhead{\asof}
}


\usepackage{multirow}

\setlength\parindent{0pt}

\newcommand{\sect}[1]{{\fontsize{17}{30}\selectfont \textbf{#1}} {\vspace{0.1cm}} \hrule {\vspace{0.3cm}}}

\newcommand{\subsect}[1]{{\fontsize{14}{24}\selectfont \textit{\textbf{#1}}} {\vspace{0.3cm}}}

\begin{document}

%%%%%%%%%%%% TITLE %%%%%%%%%%%%%
\begin{center}
{\Large \textbf{Eunsun Jou}}
\end{center}
%%%%%%%%%%%%%%%%%%%%%%%%%%%%%

\thispagestyle{firstpage}

%%%%%%%%%%%% CONTACT %%%%%%%%%%%%%
\sect{Contact}

\href{mailto:jou@mit.edu}{jou@mit.edu}

\href{https://eunsunjou.github.io}{https://eunsunjou.github.io}

D-866, 32 Vassar St, Cambridge, MA, USA 02139
%%%%%%%%%%%%%%%%%%%%%%%%%%%%%%%%

\vspace{1cm}

%%%%%%%%%%%% EDUCATION %%%%%%%%%%%%%
\sect{Education}

\subsubsection*{Degree Programs}
\begin{tabular}{p{0.1\textwidth}|p{0.9\textwidth}}
    {2024}&{\textbf{Ph.D. in Linguistics -- Massachusetts Institute of Technology}}\\
    {(expected)}&{}\\
\end{tabular}

\vspace{0.2cm}

\begin{tabular}{p{0.1\textwidth}|p{0.9\textwidth}}
	{2018}&{\textbf{M.A. in Linguistics -- Seoul National University}}\\
 \end{tabular}

\vspace{0.2cm}

\begin{tabular}{p{0.1\textwidth}|p{0.9\textwidth}}
   {2016}&{\textbf{B.A. in Linguistics -- Seoul National University}}\\
\end{tabular}

\subsubsection*{Non-Degree Programs}
\begin{tabular}{p{0.1\textwidth}|p{0.9\textwidth}}
	{2017}&{\textbf{Linguistic Institute -- Linguistic Society of America}}\\
\end{tabular}

%%%%%%%%%%%%%%%%%%%%%%%%%%%%%%%%

{\vspace{1cm}}

{\sect{Publications}}

{\subsect{Articles}}

[Under review] Honorification as Agree in Korean and Beyond.\\

{\subsect{Conference Proceedings}}

{[To Appear] An economy-based amendment to learning hidden structure with Robust Interpretive Parsing. \textit{Proceedings of the 2022 Annual Meeting on Phonology.}}\\

{2018. Embedded topicalization in Korean factive complement clauses: An experimental approach. \textit{Proceedings of Japanese/Korean Linguistics 26.}}\\

{2017. The distribution and interpretation of the Korean topic marker \textit{nun} -- a feature-inheritance approach. \textit{Proceedings of the 2017 Linguistic Society of Korea Summer Conference}. 145-156.

{\pagebreak}

\sect{Conference Presentations}

{2022. An economy-based amendment to Robust Interpretive Parsing with the GLA. Annual Meeting on Phonology.}\\

{2022. Korean addressee honorification as Cyclic Agree at Force: Comparison with Magahi. The 45th Generative Linguistics in the Old World.}\\

{2019. Embedded topicalization in Korean factive complement clauses. 93rd Annual Meeting of the Linguistic Society of America.}\\

{2018. Embedded topicalization in Korean factive clauses and islands: Experimental approach. The 26th Japanese/Korean Linguistics Conference.}\\

{2017. The distribution and interpretation of the Korean topic marker \textit{nun} -- a feature-inheritance approach. The Linguistic Society of Korea Summer Conference.}

{\vspace{1cm}}

\sect{Unpublished Manuscripts}

2022. An economy-based amendment to learning hidden structure with Robust Intepretive Parsing. Generals paper. MIT.\\

2018. Embedded Topicalization in Korean Factive Complement Clauses: An Experimental Approach. M.A. Thesis. Seoul National University. (Supervisor: Heejeong Ko)

{\vspace{1cm}}

\sect{Teaching Assistantships}

Summer 2022. Introduction to Phonology (Crete Summer School of Linguistics). Instructor: Douglas Pulleyblank.\\

Spring 2022. Language and Structure II: Syntax (MIT). Instructor: Sabine Iatridou.\\

Fall 2021. Introduction to linguistics (MIT). Instructor: Adam Albright.

{\pagebreak}

\sect{Other Services}

\subsubsection*{Massachusetts Institute of Technology}
Fall 2021 -- Spring 2022. Colloquium co-organizer

Fall 2020 -- Spring 2021. Syntax Square (Reading group) co-organizer

\subsubsection*{Seoul National University}
{2017. Research Assistant for the World-leading University Fostering Program.

\begin{itemize}[leftmargin=15pt, topsep=0pt, itemsep=0pt, parsep=0pt]
	\item{{\small Assistant to organizing committee of the 19th Seoul International Conference on Generative Grammar (SICOGG 19)}}
	\item{{\small Collected and compiled data for the departmental consulting project}}
\end{itemize}
}

\vspace{1cm}

\sect{Scholarships}

\textbf{First Year Presidential Fellowship} (2019)\\{Massachusetts Institute of Technology}\\

{\textbf{Eminence Scholarship}} (2015)\\{Seoul National University}\\

{\textbf{Merit-based Scholarship}} (2012, 2014)\\{Seoul National University}\\

{\textbf{Scholarship for Superior Academic Performance}} (2011)\\{Seoul National University}\\

\vspace{1cm}

\sect{Skills}

\subsubsection*{Languages}
\noindent{Korean (native)}, English (fluent), French (advanced), German (reading knowledge)
\subsubsection*{Computational}
\noindent{Python,} {\LaTeX}, R, Perl

\end{document}
%%% Local Variables:
%%% mode: latex
%%% TeX-master: t
%%% End:
