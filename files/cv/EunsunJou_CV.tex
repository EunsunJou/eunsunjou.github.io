% !TEX TS-program = xelatex
% !TEX encoding = UTF-8 Unicode

\documentclass[10pt]{article}
\usepackage[letterpaper, margin=1in, includehead, includefoot]{geometry}

\usepackage{enumitem}

\usepackage{fontspec}
\setmainfont{Times New Roman}

\usepackage{hyperref}
\hypersetup{
	colorlinks   = true, %Colours links instead of ugly boxes
	urlcolor     = black, %Colour for external hyperlinks
	linkcolor    = black, %Colour of internal links
	citecolor   = black %Colour of citations
}



\usepackage{datetime}

\newdateformat{monthyeardate}{%
  \monthname[\THEMONTH], \THEYEAR}

\usepackage{hyperref}

\newcommand{\asof}{\monthyeardate\today}

\usepackage{lastpage}

\usepackage{hanging}

\usepackage{fancyhdr}
\pagestyle{fancy}
\setlength{\headheight}{12pt}
\renewcommand{\headrulewidth}{0pt} % decorative line below header
\setlength{\headsep}{15pt}
\cfoot{Page {\thepage} of {\pageref*{LastPage}}} % page number
\fancyhead[L]{Eunsun Jou}
\fancyhead[R]{\asof}
\fancypagestyle{firstpage}{%
	\lhead{}
	\rhead{\asof}
}


\usepackage{multirow}

\setlength\parindent{0pt}

\newcommand{\sect}[1]{\vspace{5mm} {\fontsize{14}{21}\selectfont \textbf{#1}} {\vspace{0.1cm}} \hrule {\vspace{0.3cm}}}

\newcommand{\subsect}[1]{\vspace{3mm} {\fontsize{11}{18}\selectfont \textit{\textbf{#1}}} {\vspace{0.3cm}}}

\newcommand{\paper}[2]{\begin{tabular}{p{0.1\textwidth}p{0.9\textwidth}}{#1}&{#2}\\\end{tabular}\vspace{2mm}}

\begin{document}

%%%%%%%%%%%% TITLE %%%%%%%%%%%%%
\begin{center}
{\Large \textbf{Eunsun Jou}}
\end{center}
%%%%%%%%%%%%%%%%%%%%%%%%%%%%%

\thispagestyle{firstpage}

%%%%%%%%%%%% CONTACT %%%%%%%%%%%%%

\begin{minipage}[t]{0.5\textwidth}
\begin{flushleft}
\href{mailto:jou@mit.edu}{jou@mit.edu}

\href{https://eunsunjou.github.io}{https://eunsunjou.github.io}
\end{flushleft}
\end{minipage}
\begin{minipage}[t]{0.5\textwidth}
\begin{flushright}
MIT Department of Linguistics and Philosophy

77 Massachusetts Avenue, Room 32-D808

Cambridge, MA, USA 02139
\end{flushright}
\end{minipage}

%%%%%%%%%%%%%%%%%%%%%%%%%%%%%%%%


%%%%%%%%%%%% POSITIONS %%%%%%%%%%%%%

\sect{Academic Positions}

\begin{tabular}{p{0.1\textwidth}p{0.9\textwidth}}
{2024--2025}&{\textbf{Postdoctoral associate -- Massachusetts Institute of Technology}}\\
\end{tabular}


%%%%%%%%%%%% EDUCATION %%%%%%%%%%%%%
\sect{Education}

\subsect{Degree Programs}

\begin{tabular}{p{0.1\linewidth}p{0.9\linewidth}}
{2019--2024}&{\textbf{Ph.D. in Linguistics -- Massachusetts Institute of Technology}}\\
{}&{Thesis: {\textit{Structural case on adjuncts}}}\\
{}&{Committee: Adam Albright, Heejeong Ko, David Pesetsky, Norvin Richards}
\end{tabular}

\vspace{2mm}

\begin{tabular}{p{0.1\textwidth}p{0.9\textwidth}}
{2016--2018}&{\textbf{M.A. in Linguistics -- Seoul National University}}\\
{}&{Thesis: {\textit{Embedded topicalization in Korean factive complement clauses: An experimental approach}}}\\
                  {}&{Committee: Han-Byul Chung, Heejeong Ko, Seungho Nam}\\
 \end{tabular}

\vspace{2mm}

\begin{tabular}{p{0.1\textwidth}p{0.9\textwidth}}
{2011--2016}&{\textbf{B.A. in Linguistics -- Seoul National University}}\\
\end{tabular}

\vspace{2mm}

\subsect{Non-Degree Programs}

\begin{tabular}{p{0.1\textwidth}p{0.9\textwidth}}
{2022}&{{\textbf{Crete Summer School of Linguistics.}} Rethymno, Greece.}\\
{2017}&{\textbf{Linguistic Institute -- Linguistic Society of America.} Lexington, KY, USA.}\\
\end{tabular}

%%%%%%%%%%%%%%%%%%%%%%%%%%%%%%%%

{\sect{Publications}}

{\subsect{Articles}}

\paper{2024}{Honorification as Agree in Korean and Beyond. {\textit{Glossa: a journal of general linguistics}} 9(1). {\href{https://doi.org/10.16995/glossa.9565}{https://doi.org/10.16995/glossa.9565}}}

{\subsect{Conference Proceedings}}

\paper{2024}{Positional restriction on case assignment: Evidence from Korean nominal adverbials. In Erdene-Ochir Tumen-Ochir, Julia Sinitsyna, and Shigeru Miyagawa (eds.), {\textit{Proceedings of the 17th Workshop on Altaic Formal Linguistics (WAFL 17)}}: 81--92}

\paper{2023}{An economy-based amendment to learning hidden structure with Robust Interpretive Parsing. In Noah Elkins, Bruce Hayes, Jinyoung Jo and Jian-Leat Siah (eds.), \textit{Supplemental Proceedings of the 2022 Annual Meeting on Phonology.} {\href{https://doi.org/10.3765/amp.v10i0.5420}{https://doi.org/10.3765/amp.v10i0.5420}}}

\paper{2020}{Embedded topicalization in Korean factive complement clauses: An experimental approach.
In Shoichi Iwasaki, Susan Strauss, Shin Fukuda, Sun-Ah Jun, Sung-Ock Sohn, and Kie Zuraw (eds.), \textit{Japanese/Korean Linguistics Volume 26}: 213--223.}

\paper{2017}{The distribution and interpretation of the Korean topic marker \textit{nun} -- a feature-inheritance approach. In \textit{Proceedings of the 2017 Linguistic Society of Korea Summer Conference}: 145-156.}


%%%%%%%%%%%%%%%%%%%%%%%%%%%%%%%%

{\sect{Presentations}}

\paper{Mar. 2025}{[{\textit{Upcoming}}] A successive-cyclic dependent case account of adverbial case alternation in Korean passives. The Workshop on Theoretical East Asian Linguistics 14 (TEAL 14).}

\paper{Jan. 2025}{[{\textit{Upcoming}}] Korean nonactive suffixes {\textit{HI}} and {\textit{eci}} are realizations of little {\textit{v}}. 2025 Linguistic Society of America (LSA) Annual Meeting.}

\paper{2023}{Positional restriction on case assignment: Evidence from Korean nominal adverbials. Workshop on Altaic Formal Linguistics 17 (WAFL 17).}

\paper{2023}{Case marking of Korean nominal adverbials correlates with subject position. The 25th Seoul International Conference on Generative Grammar (SICOGG 25).}

\paper{2023}{Case marking of Korean nominal adverbials correlates with subject position. Syntax Seminar at Seoul National University.}

\paper{2022}{An economy-based amendment to Robust Interpretive Parsing with the GLA. Annual Meeting on Phonology (AMP 2022).}

\paper{2022}{Korean addressee honorification as Cyclic Agree at Force: Comparison with Magahi. The 45th Generative Linguistics in the Old World (GLOW 45).}

\paper{2019}{Embedded topicalization in Korean factive complement clauses. 93rd Annual Meeting of the Linguistic Society of America (LSA).}

\paper{2018}{Embedded topicalization in Korean factive clauses and islands: Experimental approach. The 26th Japanese/Korean Linguistics Conference (J/K 26).}

\paper{2017}{The distribution and interpretation of the Korean topic marker \textit{nun} -- a feature-inheritance approach. The Linguistic Society of Korea Summer Conference.}


\sect{Teaching Assistantships}

\paper{2023}{Introduction to Linguistics (MIT 24.900) \newline Instructor: Donca Steriade}

\paper{2022}{Introduction to Phonology (Crete Summer School of Linguistics) \newline  Instructor: Douglas Pulleyblank}

\paper{2022}{Language and Structure II: Syntax (MIT 24.902) \newline Instructor: Sabine Iatridou}

\paper{2021}{Introduction to Linguistics (MIT 24.900) \newline Instructor: Adam Albright}


\sect{Services}

\subsect{Academic Services}

\vspace{-0.75cm}

\subsubsection*{Ad-hoc reviewing for journals}

 {\textit{Linguistic Inquiry}}, {\textit{Glossa}}, {\textit{Journal of East Asian Linguistics}}, {\textit{Language Research}}

\subsubsection*{Conference reviews}

CLS (2024), NELS (2023)

\vspace{\baselineskip}

\subsect{Departmental Services}

\vspace{-0.75cm}

\subsubsection*{Massachusetts Institute of Technology}
Fall 2021 -- Spring 2022. Colloquium co-organizer

Fall 2020 -- Spring 2021. Syntax Square (Reading group) co-organizer

\subsubsection*{Seoul National University}
{2017. Research Assistant for the World-leading University Fostering Program.

\begin{itemize}[leftmargin=15pt, topsep=0pt, itemsep=0pt, parsep=0pt]
	\item{{\small Assistant to organizing committee of the 19th Seoul International Conference on Generative Grammar (SICOGG)}}
	\item{{\small Collected and compiled data for the departmental consulting project}}
\end{itemize}
}

\sect{Skills}

\subsubsection*{Languages}
\noindent{Korean (native)}, English (fluent), French (advanced), German (beginner)
\subsubsection*{Computational}
\noindent{Python,} {\LaTeX}, R, Perl

\end{document}
%%% Local Variables:
%%% mode: latex
%%% TeX-master: t
%%% End:
